\documentclass[uplatex,dvipdfmx]{jsarticle}

\usepackage{amsmath,amssymb, amscd}
\usepackage{dynkin-diagrams}

%%newtheorems

\newtheorem{definition}{Definition}[section]
\newtheorem{theorem}{Theorem}[section]
\newtheorem{proposition}{Proposition}[section]
\newtheorem{lemma}{Lemma}[section]
\newtheorem{corollary}{Corollary}[section]
\newtheorem{example}{Example}[section]
\newtheorem{remark}{Remark}[section]

%%newcomands

\newcommand{\defeq}{\mathrel{\mathop:}=}

\newcommand{\lie}{\mathrm{Lie}}

\newcommand{\Sym}{\mathrm{Sym}}
\newcommand{\ch}[1]{\mathrm{ch}\left( {#1} \right)}
\newcommand{\td}[1]{\mathrm{td}\left( {#1} \right)}

\begin{document}


\section{Elliptic genus}

The elliptic genus of an complex manifold 
$M$
of dimension 
$d$
is defined as below:

\begin{equation}
    \chi(M, q, y)
    =
    \int_M
    \ch{\mathcal{ELL}_{q,y}}
    \td{M}
    \in
    \mathbb{C}[ q, y ]
\end{equation}

where

\begin{align}
    \mathcal{ELL}_{q, y}
    &\defeq
    y^{- \frac{d}{2}}
    \bigotimes_{n \geq 0}^{\infty}
    \left( 
        \wedge_{-yq^{n-1}} T_M^*
        \otimes
        \wedge_{-y^{-1}q^n} T_M
        \otimes
        \Sym_{q^n} T_M^*
        \otimes
        \Sym_{q^n}T_M
    \right), \\
    \Sym_q V
    &\defeq
    \bigoplus_{n \geq 0} ^{\infty}
    \left( 
        \Sym^n V
    \right)
    \cdot
    q^n, \\
    \wedge_q V
    &\defeq
    \bigoplus_{n \geq 0}^{\infty}
    \left( 
        \wedge^n V
    \right)
    \cdot
    q^n.
\end{align}

\section{Chern character}

Let 
$V \to M$ 
be a complex vector bundle of rank 
$r$
and let
$x_1, \ldots x_r$
be the chern roots of
$V$.

Then, the chern character 
$\ch{V}$ 
of 
$V$ 
is defined as

\begin{equation}
    \ch{V} 
    = 
    \sum_{i=1}^{r}
    e^{x_i}.
\end{equation}


\section{Todd class}

First, we define a formal power sereis:

\begin{equation}
    \mathrm{Q}(x)
    =
    \frac{x}{1 - e^x}.
\end{equation}

Let 
$x_1, \ldots x_d$ 
be the chern roots of 
$T_M$.

Then, we can define

\begin{equation}
    \td{M}
    =
    \prod_{i=1}^{d}
    \mathrm{Q}(x_i).
\end{equation}


\section{Parabolic subgroup}

Let
$G$
be a compact connected Lie group.

\begin{definition}
    A subgroup
    $P$
    of
    the complexification $G_\mathbb{C}$
    is \textit{parabolic} if it contains the Borel subgroup of
    $G$.
\end{definition}

Let
$\mathcal{S}_{\mathfrak{p}}$
be a subset of the set of simple roots
$\mathcal{S}$
of 
$G$.
We call this the \textit{set of uncrossed nodes}.
Then, we will consider a parabolic subgroup
$P$
of
$G$
which corresponds to this Lie subalgebra:

\begin{align}
    \mathfrak{p} &\defeq 
        \mathfrak{l} 
        \oplus 
        \mathfrak{n}, \\
    \mathfrak{l} &\defeq
        \mathfrak{h}
        \oplus
        \bigoplus_{\alpha \in (\mathrm{span} \mathcal{S}_{\mathfrak{p}}) \cap \Delta}
            \mathfrak{g}_\alpha, \\
    \mathfrak{n} &\defeq
        \bigoplus_{\alpha \in \Delta^+ \setminus (\mathrm{span} \mathcal{S}_{\mathfrak{p}})}
            \mathfrak{g}_\alpha.
\end{align}

We say that
$\mathfrak{l}$
is the \textit{Levi part} and
$\mathfrak{n}$
is the \textit{nilpotent part} of
$\mathfrak{p}$.

\begin{example}
    If
    $G$
    is
    $A_3$-type
    and let the uncrossed nodes be
    $\mathcal{S}_{\mathfrak{p}} = \left\{ \alpha_2, \alpha_3 \right\}$,
    then we will figure the crossed dynkin diagram of
    $P$
    as this:
    \begin{equation}
        \dynkin[label]{A}{x**}.
    \end{equation}

    The quotient
    $G/P$
    is isomorphic to
    $\mathbb{P}^3$.
\end{example}

\begin{example}
    If
    $G$
    is
    $A_5$-type
    and let the uncrossed nodes be
    $\mathcal{S}_{\mathfrak{p}} = \left\{ \alpha_1, \alpha_3, \alpha_4, \alpha_5 \right\}$,
    then we will figure the crossed dynkin diagram of
    $P$
    as this:
    \begin{equation}
        \dynkin[label]{A}{*x***}.
    \end{equation}

    In this case, the quotient
    $G/P$
    is isomorphic to
    $\mathrm{Gr}\left( 2, 6 \right)$.
\end{example}

Let
$H$
be the intesection of
$G$
and
$P$.
The Weyl group 
$W_H$
of
$H$
equals the subgroup generated by the reflections associated to the uncrossed simple roots
$\mathcal{S}_\mathfrak{p}$,
which also eqauls to the Weyl group of the Levi part
$\mathfrak{l}$(The Levi part is a semisimple Lie algebra).

\begin{example}
    Let 
    $G$ 
    be a 
    $A_5$-group
    and 
    $\mathcal{S}_{\mathfrak{p}} =$    
    $\left\{ \alpha_1, \alpha_3, \alpha_4, \alpha_5 \right\}$.
    \begin{equation*}
        \dynkin[label]{A}{*x***}
    \end{equation*}
    The Levi part
    $\mathfrak{l}$
    is isomorphic to
    $\mathfrak{sl}_2 \times \mathbb{C} \times \mathfrak{sl}_4$
    and
    $W_H \subseteq W_G$
    is isomorphic to
    $\mathfrak{S}_2 \times \mathfrak{S}_4 \subseteq \mathfrak{S}_6$
\end{example}

\section{Equivariant integration}

Let 
$G$ 
be a compact connected Lie group with a maximal torus
$T$
and let
$P$
be a parabolic subgroup of the complexification 
$G_{\mathbb{C}}$,
corresponding to a subgroup
$H$
of
$G$.

We call the inclusions
$j \colon T \hookrightarrow G$
and
$k \colon H \hookrightarrow G$.

Now, we can calculate the equivariant integration for 
$f \in \mathrm{H}^*_G(G/H)$ 
as below:

\begin{equation}
    \mathrm{B}j^*\left( 
        \mathrm{B}k_! \left( 
            f
        \right)
    \right)
    =
    \sum_{
        [w] \in W_G / W_H
    }
    \frac{f^w}{\prod_{\alpha \in \mathcal{R}} w \cdot \alpha}
\end{equation}

where

\begin{equation}
    \mathcal{R}
    =
    \Delta_G^- \setminus \Delta_H^-
    =
    \Delta_G^- \setminus (\mathrm{span} \mathcal{S}_\mathfrak{p}).
\end{equation}

\begin{equation}
    \begin{CD}
        \mathrm{H}^*_G(G/H) @>{\mathrm{B}k_!}>> \mathrm{H}_G\left( pt \right) \\
        @VVV    @V{\mathrm{B}j^*}VV \\
        \mathrm{H}^*_T(G/H) @>{\pi_!}>> \mathrm{H}^*_T(pt) 
     \end{CD}
\end{equation}

\end{document}