\documentclass[uplatex,dvipdfmx]{jsarticle}

\usepackage{amsmath,amssymb, amscd}
\usepackage{dynkin-diagrams}
\usepackage{enumerate}

%%newtheorems

\newtheorem{definition}{Definition}[section]
\newtheorem{theorem}{Theorem}[section]
\newtheorem{proposition}{Proposition}[section]
\newtheorem{lemma}{Lemma}[section]
\newtheorem{corollary}{Corollary}[section]
\newtheorem{example}{Example}[section]
\newtheorem{remark}{Remark}[section]

%%newcomands

\newcommand{\defeq}{\mathrel{\mathop:}=}

\newcommand{\lie}{\mathrm{Lie}}
%%\newcommand\span{\operatorname{span}}

\newcommand{\Sym}{\mathrm{Sym}}
\newcommand{\ch}[1]{\mathrm{ch}\left( {#1} \right)}
\newcommand{\td}[1]{\mathrm{td}\left( {#1} \right)}

\begin{document}


\section{Elliptic genus}

The elliptic genus of an complex manifold 
$M$
of dimension 
$d$
is defined as below:

\begin{equation}
    \chi(M, q, y)
    =
    \int_M
    \ch{\mathcal{ELL}_{q,y}}
    \td{M}
    \in
    \mathbb{C}[ q, y ]
\end{equation}

where

\begin{align}
    \mathcal{ELL}_{q, y}
    &\defeq
    y^{- \frac{d}{2}}
    \bigotimes_{n \geq 0}^{\infty}
    \left( 
        \wedge_{-yq^{n-1}} T_M^*
        \otimes
        \wedge_{-y^{-1}q^n} T_M
        \otimes
        \Sym_{q^n} T_M^*
        \otimes
        \Sym_{q^n}T_M
    \right), \\
    \Sym_q V
    &\defeq
    \bigoplus_{n \geq 0} ^{\infty}
    \left( 
        \Sym^n V
    \right)
    \cdot
    q^n, \\
    \wedge_q V
    &\defeq
    \bigoplus_{n \geq 0}^{\infty}
    \left( 
        \wedge^n V
    \right)
    \cdot
    q^n.
\end{align}

\section{Chern character}

Let 
$V \to M$ 
be a complex vector bundle of rank 
$r$
and let
$x_1, \ldots x_r$
be the chern roots of
$V$.

Then, the chern character 
$\ch{V}$ 
of 
$V$ 
is defined as

\begin{equation}
    \ch{V} 
    = 
    \sum_{i=1}^{r}
    e^{x_i}.
\end{equation}


\section{Todd class}

First, we define a formal power sereis:

\begin{equation}
    \mathrm{Q}(x)
    =
    \frac{x}{1 - e^x}.
\end{equation}

Let 
$x_1, \ldots x_d$ 
be the chern roots of 
$T_M$.

Then, we can define

\begin{equation}
    \td{M}
    =
    \prod_{i=1}^{d}
    \mathrm{Q}(x_i).
\end{equation}


\section{Parabolic subgroup}

Let
$G$
be a reductive complex Lie group.

\begin{definition}
    A subgroup
    $P$
    of
    $G$
    is \textit{parabolic} if it contains the Borel subgroup of
    $G$.
\end{definition}

Let
$\mathcal{S}_{\mathfrak{p}}$
be a subset of the set of simple roots
$\mathcal{S}$
of 
$G$.
We call this the \textit{set of uncrossed nodes}.
Then, we will consider a parabolic subgroup
$P$
of
$G$
which corresponds to this Lie subalgebra:

\begin{align}
    \mathfrak{p} &\defeq 
        \mathfrak{l} 
        \oplus 
        \mathfrak{n}, \\
    \mathfrak{l} &\defeq
        \mathfrak{h}
        \oplus
        \bigoplus_{\alpha \in (\operatorname{span} \mathcal{S}_{\mathfrak{p}}) \cap \Delta}
            \mathfrak{g}_\alpha, \\
    \mathfrak{n} &\defeq
        \bigoplus_{\alpha \in \Delta^+ \setminus (\operatorname{span} \mathcal{S}_{\mathfrak{p}})}
            \mathfrak{g}_\alpha.
\end{align}

We say that
$\mathfrak{l}$
is the \textit{Levi part} and
$\mathfrak{n}$
is the \textit{nilpotent part} of
$\mathfrak{p}$.

\begin{example}
    If
    $G$
    is
    $A_3$-type
    and let the uncrossed nodes be
    $\mathcal{S}_{\mathfrak{p}} = \left\{ \alpha_2, \alpha_3 \right\}$,
    then we will figure the crossed dynkin diagram of
    $P$
    as this:
    \begin{equation}
        \dynkin[label]{A}{x**}.
    \end{equation}

    The quotient
    $G/P$
    is isomorphic to
    $\mathbb{P}^3$.
\end{example}

\begin{example}
    If
    $G$
    is
    $A_5$-type
    and let the uncrossed nodes be
    $\mathcal{S}_{\mathfrak{p}} = \left\{ \alpha_1, \alpha_3, \alpha_4, \alpha_5 \right\}$,
    then we will figure the crossed dynkin diagram of
    $P$
    as this:
    \begin{equation}
        \dynkin[label]{A}{*x***}.
    \end{equation}

    In this case, the quotient
    $G/P$
    is isomorphic to
    $\mathrm{Gr}\left( 2, 6 \right)$.
\end{example}

Let
$H$
be a real part of
$P$.
The Weyl group 
$W_H$
of
$H$
equals the subgroup generated by the reflections associated to the uncrossed simple roots
$\mathcal{S}_\mathfrak{p}$,
which also eqauls to the Weyl group of the Levi part
$\mathfrak{l}$(The Levi part is a reductive Lie algebra).

\begin{example}
    Let 
    $G$ 
    be a 
    $A_5$-group
    and 
    $\mathcal{S}_{\mathfrak{p}} =$    
    $\left\{ \alpha_1, \alpha_3, \alpha_4, \alpha_5 \right\}$.
    \begin{equation*}
        \dynkin[label]{A}{*x***}
    \end{equation*}
    The Levi part
    $\mathfrak{l}$
    is isomorphic to
    $\mathfrak{sl}_2 \times \mathbb{C} \times \mathfrak{sl}_4$
    and
    $W_H \subseteq W_G$
    is isomorphic to
    $\mathfrak{S}_2 \times \mathfrak{S}_4 \subseteq \mathfrak{S}_6$
\end{example}


\begin{example}
    We assume that
    $G$
    is
    $A_n$-type
    and
    $\mathcal{S}_\mathfrak{p}$
    is any subset of 
    $\left\{\alpha_1, \ldots, \alpha_n \right\}$.

    We denote the maximal connected intervals consisting of uncrossed nodes by
    \begin{equation}
        \{[n(1), n(1) + l(1) - 1], \ldots, [n(m), n(m) + l(m) - 1]\}.
    \end{equation}
    Then,
    \begin{equation}
        G/P \cong \mathrm{SU}(n+1)/\mathrm{SU}
    \end{equation}
\end{example}

\section{Chern roots of equivariant vector bundles}


Let
$G$
be a reductive complex Lie group and
$\Lambda$
be the weight lattice of
$G$
with fundamental weights
$\left\{ \omega_1, \omega_2, \ldots, \omega_r \right\}$.
Let
$P$
be a parabolic subgroup of
$G$.

\begin{definition}
    Let 
    $\lambda = \sum_{i = 1}^r \lambda_i \omega_i$ 
    be a weight of
    $G$.
    \begin{enumerate}
        \item
            $\lambda$
            is
            \textit{integral} 
            if
            $\lambda_i$
            is integral for any
            $i = 1, \ldots, r$.
        \item 
            $\lambda$
            is
            $\mathfrak{p}$\textit{-dominant} 
            if
            $\lambda$
            is integral and
            $\lambda_i$
            is non-negative for any
            $i$
            such that
            $\alpha_i \in \mathcal{S}_\mathfrak{p}$.
        \item 
            $\lambda$
            is
            $\mathfrak{g}$\textit{-dominant} 
            if
            $\lambda$
            is integral and
            $\lambda_i$
            is non-negative for any
            $i = 1, \ldots, r$.
    \end{enumerate}
\end{definition}

\subsection{The weight set of irreducible representations of $P$}

For an integral
$\mathfrak{p}$-dominant
weight
$\lambda \in \Lambda$,
we get a irreducible representation
$V_\lambda^P$.
We can calculate the set of its weights which is denoted by
$\Delta(V_\lambda^P)$ as below:

\underline{Step 1:}
Make a list of maximal connected dynkin subdiagrams consisting of uncrossed nodes
\begin{equation}
    \left\{ 
        \mathcal{S}_1, 
        \ldots, 
        \mathcal{S}_m 
    \right\}
    =
    \left\{
        \left\{ \alpha_{n(1)}, \alpha_{n(1)+1}, \ldots \right\},
        \ldots, 
        \left\{ \alpha_{n(m)}, \alpha_{n(m)+1}, \ldots \right\} 
    \right\}.
\end{equation}

\underline{Step 2:}
For each subdiagram 
$\mathcal{S}_i$, 
calculate the weight set of the highest weight representation
associated to the weight
$\lambda_i 
\defeq
\sum_{k = 0}^{\# \mathcal{S}_i - 1}
\lambda_{n(i) + k} \omega_{n(i) + k}$
of 
$\mathfrak{g}_i
\defeq
(\operatorname{span} \mathcal{S}_i)_\mathbb{C} 
\oplus 
\bigoplus_{\alpha \in (\operatorname{span} \mathcal{S}_i)\cap \Delta}
\mathfrak{g}_\alpha$ which can be described as below:
\begin{equation}
    \Delta\left( 
        V
        ^{\mathfrak{g}_i}
        _{\lambda_i}
    \right)
    =
    \left\{ 
        \lambda_i - \sum_{\alpha_k \in \mathcal{S}_i} n_k^{(j)} \alpha_k
    \right\}
    _{j = 1}
    ^{\operatorname{dim}
        V
        ^{\mathfrak{g}_i}
        _{\lambda_i}}.
\end{equation}

\underline{Step 3:}
Then, the weight set 
$\Delta(V_\lambda^P)$
is
\begin{equation}
    \left\{ 
        \lambda 
        - 
        \sum_{i = 1}^m
        \sum_{\alpha_k \in \mathcal{S}_i} 
        n_k^{(j_i)} \alpha_k
    \right\}_{\vec{j} \in J}
\end{equation}
where
\begin{equation}
    \vec{j}
    =
    \left( j_1, \ldots, j_m \right)
    \in
    J
    =
    \prod_{i = 1}^m
    \left\{ 1, \ldots, \operatorname{dim} V^{\mathfrak{g}_i}_{\lambda_i} \right\}.
\end{equation}

\subsection{Chern roots of equivariant vector bundles}

For an integral
$\mathfrak{p}$-dominant
weight
$\lambda \in \Lambda$,
we define an equivariant vector bundle 
$\mathcal{E}_\lambda$
over
$G/P$
by
\begin{equation}
    \mathcal{E}_\lambda
    \defeq
    \left( 
        G
        \times_P
        V_\lambda
     \right)^{\vee}.
\end{equation}

The set of chern roots of
$\mathcal{E}_\lambda$
equals to
\begin{equation}
    \left\{ 
        \sum_{i = 1}^r
        \lambda_i
        x_i
        \middle|
        \lambda \in \Delta((V_\lambda^P)^{\vee})
    \right\}
\end{equation}

where
$x_i$
is the corresponding element of
$\omega_i$
in
$\mathrm{H}_G(G/B)$
along the isomorphism
$\mathrm{H}_G(G/B) \cong \Sym^* \Lambda$.

\subsection{Chern roots of the tangent bundle of G/P}

The tangent bundle 
$T_{G/P}$
is isomorphic to
$G \times_P (\mathfrak{g}/\mathfrak{p})$
where
$\mathfrak{g}/\mathfrak{p}$
is a 
$P$-vector space 
with the adjoint action.
Thus, the set of chern roots is
\begin{equation}
    \left\{ 
        \sum_{i = 1}^r
        \lambda_i
        x_i
        \middle|
        \lambda \in \Delta(\mathfrak{g}/\mathfrak{p})
        =
        \Delta^-_G 
        \setminus
        \operatorname{span} \mathcal{S}_\mathfrak{p}
    \right\}.
\end{equation}

\section{Equivariant integration}

Let 
$G$ 
be a compact connected Lie group with a maximal torus
$T$
and let
$P$
be a parabolic subgroup of the complexification 
$G_{\mathbb{C}}$,
corresponding to a subgroup
$H$
of
$G$.

We call the inclusions
$j \colon T \hookrightarrow G$
and
$k \colon H \hookrightarrow G$.

Now, we can calculate the equivariant integration for 
$f \in \mathrm{H}^*_G(G/H)$ 
as below:

\begin{equation}
    \mathrm{B}j^*\left( 
        \mathrm{B}k_! \left( 
            f
        \right)
    \right)
    =
    \sum_{
        [w] \in W_G / W_H
    }
    \frac{f^w}{\prod_{\alpha \in \mathcal{R}} w \cdot \alpha}
\end{equation}

where

\begin{equation}
    \mathcal{R}
    =
    \Delta_G^- \setminus \Delta_H^-
    =
    \Delta_G^- \setminus (\operatorname{span} \mathcal{S}_\mathfrak{p}).
\end{equation}

\begin{equation}
    \begin{CD}
        \mathrm{H}^*_G(G/H) @>{\mathrm{B}k_!}>> \mathrm{H}_G^*\left( pt \right) \\
        @VVV    @V{\mathrm{B}j^*}VV \\
        \mathrm{H}^*_T(G/H) @>{\pi_!}>> \mathrm{H}^*_T(pt) \cong \mathrm{H}^*_G(G_\mathbb{C}/B)
     \end{CD}
\end{equation}

\section{Construction of a Calabi-Yau manifold}

Let
$G$
be a complex Lie group of type
$F_4$
and let
$\lambda = \left( 0, 1, 1, 0 \right)$
be a weight
of
$G$.
Let us consider two parabolic Lie group
$P_1$
and
$P_2$
corresponding to these dynkin diagrams.
\begin{align}
    P_1\colon \dynkin[label]{F}{**x*} \\
    P_2\colon \dynkin[label]{F}{*x**}
\end{align}


\begin{align}
    \left( \mathfrak{g}/\mathfrak{p_1} \right)^\vee
    &\cong
    V^{P_1}_{(1, 0, -1, 1)}
    +
    V^{P_1}_{(0, 1, -2, 2)}
    +
    V^{P_1}_{(0, 0, 0, 1)}
    +
    V^{P_1}_{(1, 0, 0, 0)} \\
    \left( \mathfrak{g}/\mathfrak{p_2} \right)^\vee
    &\cong
    V^{P_2}_{(1, -1, 0, 2)}
    +
    V^{P_2}_{(0, -1, 2, 0)}
    +
    V^{P_2}_{(1, 0, 0, 0)}
\end{align}


\begin{equation}
    \begin{alignedat}{2}
        \Delta\left( V^{P_1}_{(1, 0, -1, 1)} \right)
        =
        \{ 
            &\lambda_{1, 1} = (1, 0, -1, 1),\\
            &\lambda_{1, 1} - \alpha_1,\\
            &\lambda_{1, 1} - \alpha_1 &- \alpha_2,\\
            &\lambda_{1, 1} &&- \alpha_4,\\
            &\lambda_{1, 1} - \alpha_1 &&- \alpha_4,\\
            &\lambda_{1, 1} - \alpha_1 &- \alpha_2 &- \alpha_4\\
        \}
    \end{alignedat}
\end{equation}

\begin{equation}
    \begin{alignedat}{2}
        \Delta\left( V^{P_1}_{(0, 1, -2, 2)} \right)
        =
        \{
            &\lambda_{1,2} = (0, 1, -2, 2),\\
            &\lambda_{1,2} &- \alpha_2,\\
            &\lambda_{1,2} - \alpha_1 &- \alpha_2,\\
            &\lambda_{1,2} &&- \alpha_4,\\
            &\lambda_{1,2} &- \alpha_2 &- \alpha_4,\\
            &\lambda_{1,2} - \alpha_1 &- \alpha_2 &- \alpha_4,\\
            &\lambda_{1,2} &&- 2\alpha_4,\\
            &\lambda_{1,2} &- \alpha_2 &- 2\alpha_4,\\
            &\lambda_{1,2} - \alpha_1 &- \alpha_2 &- 2\alpha_4\\
            \}
    \end{alignedat}
\end{equation}

\begin{equation}
    \begin{alignedat}
        \Delta\left( V^{P_1}_{(0, 0, 0, 1)} \right)
        =
        \{
            &\lambda_{1,3} = (0, 0, 0, 1), \\
            &\lambda_{1,3} - \alpha_4\\
        \}
    \end{alignedat}
\end{equation}

\begin{equation}
    \begin{alignedat}
        \Delta\left( V^{P_1}_{(1, 0, 0, 0)} \right)
        =
        \{
            &\lambda_{1,4} = (1, 0, 0, 0), \\
            &\lambda_{1,4} - \alpha_1,\\
            &\lambda_{1,4} - \alpha_1 - \alpha_2\\
        \}
    \end{alignedat}
\end{equation}

\begin{equation}
    \begin{alignedat}
        \Delta\left( V^{P_2}_{(1, -1, 0, 2)} \right)
        =
        \{
            &\lambda_{2,1} = (1, -1, 0, 2),\\
            &\lambda_{2,1} &&- \alpha_4,\\
            &\lambda_{2,1} &- \alpha_3 &-\alpha_4,\\
            &\lambda_{2,1} &&- 2\alpha_4,\\
            &\lambda_{2,1} &- \alpha_3 &- 2\alpha_4,\\
            &\lambda_{2,1} &- 2\alpha_3 &- 2\alpha_4,\\
            &\lambda_{2,1} - \alpha_1,\\
            &\lambda_{2,1} - \alpha_1 &&- \alpha_4,\\
            &\lambda_{2,1} - \alpha_1 &- \alpha_3 &-\alpha_4,\\
            &\lambda_{2,1} - \alpha_1 &&- 2\alpha_4,\\
            &\lambda_{2,1} - \alpha_1 &- \alpha_3 &- 2\alpha_4,\\
            &\lambda_{2,1} - \alpha_1 &- 2\alpha_3 &- 2\alpha_4\\
        \}
    \end{alignedat}
\end{equation}

\begin{equation}
    \begin{alignedat}{2}
        \Delta\left( V^{P_2}_{(0, -1, 2, 0)} \right)
        =
        \{
            &\lambda_{2,2} = (0, -1, 2, 0),\\
            &\lambda_{2,2} - \alpha_3,\\
            &\lambda_{2,2} - 2\alpha_3,\\
            &\lambda_{2,2} - \alpha_3 -\alpha_4,\\
            &\lambda_{2,2} - 2\alpha_3 - \alpha_4,\\
            &\lambda_{2,2} - 2\alpha_3 - 2\alpha_4\\
        \}
    \end{alignedat}
\end{equation}

\begin{equation}
    \begin{alignedat}
        \Delta\left( V^{P_2}_{(1, 0, 0, 0)} \right)
        =
        \{
            &\lambda_{2,3} = (1, 0, 0, 0),\\
            &\lambda_{2,3} - \alpha_1\\
        \}
    \end{alignedat}
\end{equation}


Let
$\mathcal{E}_\lambda \to G/P$
be an equivariant vector bundle over a homogeneous space.
For a generic section
$s$
of
$\mathcal{E}_\lambda$,
the zero locus
$X \defeq s^{-1}(0)$
is a smooth complete intersection with chern class
\begin{align}
    \mathrm{c}(X)
    &=
    \frac
    {\mathrm{c}(G/P)}
    {\mathrm{c}(\mathcal{E})}\\
    &=
    \frac
    {\prod_{\mu \in \Delta(\mathfrak{g}/\mathfrak{p}^\vee)}(1+\mathrm{c}_1(\mathcal{L}_\mu))}
    {\prod_{\nu \in \Delta(V^{P\vee}_\lambda)}(1+\mathrm{c}_1(\mathcal{L}_\nu))}
\end{align}

Now, we fix
$G$,
$P_1$
and
$P_2$
as above and
$\lambda = (0, 1, 1, 0)$.
Then, the zero loci
$X_1$
and
$X_2$
are Calabi-Yau,
i.e. their
$\mathrm{c}_1$
equal to
$0$.

\begin{equation}
    \begin{alignedat}
        \Delta\left( V^{P_1}_{(0, 1, 1, 0)} \right)
        =
        \{
            &\lambda = (0, 1, 1, 0),\\
            &\lambda - \alpha_2,\\
            &\lambda - \alpha_1 - \alpha_2\\
        \}
    \end{alignedat}
\end{equation}

\begin{equation}
    \begin{alignedat}
        \Delta\left( V^{P_2}_{(0, 1, 1, 0)} \right)
        =
        \{
            &\lambda = (0, 1, 1, 0),\\
            &\lambda - \alpha_3,\\
            &\lambda - \alpha_3 - \alpha_4\\
        \}
    \end{alignedat}
\end{equation}

\begin{align}
    \mathrm{c}_1(T_{G/P_1})
    &=
    \sum_{\mu \in \Delta(\mathfrak{g}/\mathfrak{p}_1^\vee)}\mathrm{c}_1(\mathcal{L}_\mu)
    -
    \sum_{\nu \in \Delta(V^{P_1\vee}_\lambda)}\mathrm{c}_1(\mathcal{L}_\nu)\\
    &=
    0
\end{align}

The same equation for
$G/P_2$
is also true.


\section{Formula for verifications}

Let
$G$
be a reductive complex Lie group with a parabolic subgroup
$P$.
Let
$\lambda$
be a
$\mathfrak{p}$-dominant
weight.
Then, there is a dimension formula 
induced by HRR and Weyl character formula.

\begin{align}
    \chi \left( G/P, \mathcal{E}_\lambda \right)
    &=
    \int_{G/P}
    \ch{\mathcal{E}_\lambda}
    \td{G/P}\\
    &= \sum_{
        [w] \in W_G / W_P
    }
    \frac
    {\sum_{i + j = \operatorname{dim} G/P}(\mathrm{ch}_i \left( \mathcal{E}_\lambda \right)
    \mathrm{td}_j\left( G/P \right))^w}
    {\prod_{\alpha \in \mathcal{R}} w \cdot \alpha}\\
    &=
    (-1)^{l(w')}\operatorname{dim} V^G_{w' * \lambda}\\
    &=
    \frac
    {\prod_{\alpha \in \Delta_G^+}((w'*\lambda) + \rho, \alpha)}
    {\prod_{\alpha \in \Delta_G^+}(\rho, \alpha)}\\
    &=
    \frac
    {\prod_{\alpha \in \Delta_G^+}(w'(\lambda + \rho), \alpha)}
    {\prod_{\alpha \in \Delta_G^+}(\rho, \alpha)}
\end{align}

where the equivariant Chern character is 
\begin{equation}
    \ch{\mathcal{E}_\lambda} = \prod_{\nu \in \Delta(V_\lambda^\vee)}(1 + \nu),
\end{equation}

the equivariant Todd class is
\begin{equation}
    \td{G/P} = \prod_{\mu \in \Delta(\mathfrak{g}/\mathfrak{p}^\vee)}
    \frac{\mu}{1 - e^\mu},
\end{equation}
and
$w'$
is the unique(if it exist) element of Weyl group of
$P$
such that
$w' * \lambda$
is
$\mathfrak{g}$-dominant,
$l(w')$
is the length of
$w'$
and
$w *\lambda$
means the
$(-\rho)$-centered
action by
$w$,
i.e.
$w(\lambda + \rho) - \rho$.
\end{document}